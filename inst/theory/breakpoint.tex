% Created 2024-06-18 ti 13:58
% Intended LaTeX compiler: pdflatex
\documentclass[12pt]{article}

%%%% settings when exporting code %%%% 

\usepackage{listings}
\lstdefinestyle{code-small}{
backgroundcolor=\color{white}, % background color for the code block
basicstyle=\ttfamily\small, % font used to display the code
commentstyle=\color[rgb]{0.5,0,0.5}, % color used to display comments in the code
keywordstyle=\color{black}, % color used to highlight certain words in the code
numberstyle=\ttfamily\tiny\color{gray}, % color used to display the line numbers
rulecolor=\color{black}, % color of the frame
stringstyle=\color[rgb]{0,.5,0},  % color used to display strings in the code
breakatwhitespace=false, % sets if automatic breaks should only happen at whitespace
breaklines=true, % sets automatic line breaking
columns=fullflexible,
frame=single, % adds a frame around the code (non,leftline,topline,bottomline,lines,single,shadowbox)
keepspaces=true, % % keeps spaces in text, useful for keeping indentation of code
literate={~}{$\sim$}{1}, % symbol properly display via latex
numbers=none, % where to put the line-numbers; possible values are (none, left, right)
numbersep=10pt, % how far the line-numbers are from the code
showspaces=false,
showstringspaces=false,
stepnumber=1, % the step between two line-numbers. If it's 1, each line will be numbered
tabsize=1,
xleftmargin=0cm,
emph={anova,apply,class,coef,colnames,colNames,colSums,dim,dcast,for,ggplot,head,if,ifelse,is.na,lapply,list.files,library,logLik,melt,plot,require,rowSums,sapply,setcolorder,setkey,str,summary,tapply},
aboveskip = \medskipamount, % define the space above displayed listings.
belowskip = \medskipamount, % define the space above displayed listings.
lineskip = 0pt} % specifies additional space between lines in listings
\lstset{style=code-small}
%%%% packages %%%%%

\usepackage[utf8]{inputenc}
\usepackage[T1]{fontenc}
\usepackage{lmodern}
\usepackage{textcomp}
\usepackage{color}
\usepackage{graphicx}
\usepackage{grffile}
\usepackage{wrapfig}
\usepackage{rotating}
\usepackage{longtable}
\usepackage{multirow}
\usepackage{multicol}
\usepackage{changes}
\usepackage{pdflscape}
\usepackage{geometry}
\usepackage[normalem]{ulem}
\usepackage{amssymb}
\usepackage{amsmath}
\usepackage{amsfonts}
\usepackage{dsfont}
\usepackage{array}
\usepackage{ifthen}
\usepackage{hyperref}
\usepackage{natbib}
%
%%%% specifications %%%%
%
\usepackage{ifthen}
\usepackage{xifthen}
\usepackage{xargs}
\usepackage{xspace}
\RequirePackage{fancyvrb}
\DefineVerbatimEnvironment{verbatim}{Verbatim}{fontsize=\small,formatcom = {\color[rgb]{0.5,0,0}}}
\RequirePackage{colortbl} % arrayrulecolor to mix colors
\RequirePackage{setspace} % to modify the space between lines - incompatible with footnote in beamer
\renewcommand{\baselinestretch}{1.1}
\geometry{top=1cm}
\RequirePackage{colortbl} % arrayrulecolor to mix colors
\RequirePackage{pifont}
\RequirePackage{relsize}
\newcommand{\Cross}{{\raisebox{-0.5ex}%
{\relsize{1.5}\ding{56}}}\hspace{1pt} }
\newcommand{\Valid}{{\raisebox{-0.5ex}%
{\relsize{1.5}\ding{52}}}\hspace{1pt} }
\newcommand{\CrossR}{ \textcolor{red}{\Cross} }
\newcommand{\ValidV}{ \textcolor{green}{\Valid} }
\usepackage{stackengine}
\usepackage{scalerel}
\newcommand\Warning[1][3ex]{%
\renewcommand\stacktype{L}%
\scaleto{\stackon[1.3pt]{\color{red}$\triangle$}{\tiny\bfseries !}}{#1}%
\xspace
}
\hypersetup{
citecolor=[rgb]{0,0.5,0},
urlcolor=[rgb]{0,0,0.5},
linkcolor=[rgb]{0,0,0.5},
}
\RequirePackage{epstopdf} % to be able to convert .eps to .pdf image files
\RequirePackage{capt-of} %
\RequirePackage{caption} % newlines in graphics
\RequirePackage{tikz}
\definecolor{grayR}{HTML}{8A8990}
\definecolor{grayL}{HTML}{C4C7C9}
\definecolor{blueM}{HTML}{1F63B5}
\newcommand{\Rlogo}[1][0.07]{
\begin{tikzpicture}[scale=#1]
\shade [right color=grayR,left color=grayL,shading angle=60]
(-3.55,0.3) .. controls (-3.55,1.75)
and (-1.9,2.7) .. (0,2.7) .. controls (2.05,2.7)
and (3.5,1.6) .. (3.5,0.3) .. controls (3.5,-1.2)
and (1.55,-2) .. (0,-2) .. controls (-2.3,-2)
and (-3.55,-0.75) .. cycle;

\fill[white]
(-2.15,0.2) .. controls (-2.15,1.2)
and (-0.7,1.8) .. (0.5,1.8) .. controls (2.2,1.8)
and (3.1,1.2) .. (3.1,0.2) .. controls (3.1,-0.75)
and (2.4,-1.45) .. (0.5,-1.45) .. controls (-1.1,-1.45)
and (-2.15,-0.7) .. cycle;

\fill[blueM]
(1.75,1.25) -- (-0.65,1.25) -- (-0.65,-2.75) -- (0.55,-2.75) -- (0.55,-1.15) --
(0.95,-1.15)  .. controls (1.15,-1.15)
and (1.5,-1.9) .. (1.9,-2.75) -- (3.25,-2.75)  .. controls (2.2,-1)
and (2.5,-1.2) .. (1.8,-0.95) .. controls (2.6,-0.9)
and (2.85,-0.35) .. (2.85,0.2) .. controls (2.85,0.7)
and (2.5,1.2) .. cycle;

\fill[white]  (1.4,0.4) -- (0.55,0.4) -- (0.55,-0.3) -- (1.4,-0.3).. controls (1.75,-0.3)
and (1.75,0.4) .. cycle;

\end{tikzpicture}
}
\RequirePackage{enumitem} % to be able to convert .eps to .pdf image files
\definecolor{light}{rgb}{1, 1, 0.9}
\definecolor{lightred}{rgb}{1.0, 0.7, 0.7}
\definecolor{lightblue}{rgb}{0.0, 0.8, 0.8}
\newcommand{\darkblue}{blue!80!black}
\newcommand{\darkgreen}{green!50!black}
\newcommand{\darkred}{red!50!black}
\usepackage{mdframed}
\newcommand{\first}{1\textsuperscript{st} }
\newcommand{\second}{2\textsuperscript{nd} }
\newcommand{\third}{3\textsuperscript{rd} }
\RequirePackage{amsmath}
\RequirePackage{algorithm}
\RequirePackage[noend]{algpseudocode}
\RequirePackage{dsfont}
\RequirePackage{amsmath,stmaryrd,graphicx}
\RequirePackage{prodint} % product integral symbol (\PRODI)
\newcommand\defOperator[7]{%
\ifthenelse{\isempty{#2}}{
\ifthenelse{\isempty{#1}}{#7{#3}#4}{#7{#3}#4 \left#5 #1 \right#6}
}{
\ifthenelse{\isempty{#1}}{#7{#3}#4_{#2}}{#7{#3}#4_{#1}\left#5 #2 \right#6}
}
}
\newcommand\defUOperator[5]{%
\ifthenelse{\isempty{#1}}{
#5\left#3 #2 \right#4
}{
\ifthenelse{\isempty{#2}}{\underset{#1}{\operatornamewithlimits{#5}}}{
\underset{#1}{\operatornamewithlimits{#5}}\left#3 #2 \right#4}
}
}
\newcommand{\defBoldVar}[2]{
\ifthenelse{\equal{#2}{T}}{\boldsymbol{#1}}{\mathbf{#1}}
}
\newcommandx\Cov[2][1=,2=]{\defOperator{#1}{#2}{C}{ov}{\lbrack}{\rbrack}{\mathbb}}
\newcommandx\Esp[2][1=,2=]{\defOperator{#1}{#2}{E}{}{\lbrack}{\rbrack}{\mathbb}}
\newcommandx\Prob[2][1=,2=]{\defOperator{#1}{#2}{P}{}{\lbrack}{\rbrack}{\mathbb}}
\newcommandx\Qrob[2][1=,2=]{\defOperator{#1}{#2}{Q}{}{\lbrack}{\rbrack}{\mathbb}}
\newcommandx\Var[2][1=,2=]{\defOperator{#1}{#2}{V}{ar}{\lbrack}{\rbrack}{\mathbb}}
\newcommandx\Binom[2][1=,2=]{\defOperator{#1}{#2}{B}{}{(}{)}{\mathcal}}
\newcommandx\Gaus[2][1=,2=]{\defOperator{#1}{#2}{N}{}{(}{)}{\mathcal}}
\newcommandx\Wishart[2][1=,2=]{\defOperator{#1}{#2}{W}{ishart}{(}{)}{\mathcal}}
\newcommandx\Likelihood[2][1=,2=]{\defOperator{#1}{#2}{L}{}{(}{)}{\mathcal}}
\newcommandx\likelihood[2][1=,2=]{\defOperator{#1}{#2}{\ell}{}{(}{)}{}}
\newcommandx\Information[2][1=,2=]{\defOperator{#1}{#2}{I}{}{(}{)}{\mathcal}}
\newcommandx\Score[2][1=,2=]{\defOperator{#1}{#2}{S}{}{(}{)}{\mathcal}}
\newcommandx\Vois[2][1=,2=]{\defOperator{#1}{#2}{V}{}{(}{)}{\mathcal}}
\newcommandx\IF[2][1=,2=]{\defOperator{#1}{#2}{IF}{}{(}{)}{\mathcal}}
\newcommandx\Ind[1][1=]{\defOperator{}{#1}{1}{}{(}{)}{\mathds}}
\newcommandx\Max[2][1=,2=]{\defUOperator{#1}{#2}{(}{)}{min}}
\newcommandx\Min[2][1=,2=]{\defUOperator{#1}{#2}{(}{)}{max}}
\newcommandx\argMax[2][1=,2=]{\defUOperator{#1}{#2}{(}{)}{argmax}}
\newcommandx\argMin[2][1=,2=]{\defUOperator{#1}{#2}{(}{)}{argmin}}
\newcommandx\cvD[2][1=D,2=n \rightarrow \infty]{\xrightarrow[#2]{#1}}
\newcommandx\Hypothesis[2][1=,2=]{
\ifthenelse{\isempty{#1}}{
\mathcal{H}
}{
\ifthenelse{\isempty{#2}}{
\mathcal{H}_{#1}
}{
\mathcal{H}^{(#2)}_{#1}
}
}
}
\newcommandx\dpartial[4][1=,2=,3=,4=\partial]{
\ifthenelse{\isempty{#3}}{
\frac{#4 #1}{#4 #2}
}{
\left.\frac{#4 #1}{#4 #2}\right\rvert_{#3}
}
}
\newcommandx\dTpartial[3][1=,2=,3=]{\dpartial[#1][#2][#3][d]}
\newcommandx\ddpartial[3][1=,2=,3=]{
\ifthenelse{\isempty{#3}}{
\frac{\partial^{2} #1}{\partial #2^2}
}{
\frac{\partial^2 #1}{\partial #2\partial #3}
}
}
\newcommand\Real{\mathbb{R}}
\newcommand\Rational{\mathbb{Q}}
\newcommand\Natural{\mathbb{N}}
\newcommand\trans[1]{{#1}^\intercal}%\newcommand\trans[1]{{\vphantom{#1}}^\top{#1}}
\newcommand{\independent}{\mathrel{\text{\scalebox{1.5}{$\perp\mkern-10mu\perp$}}}}
\newcommand\half{\frac{1}{2}}
\newcommand\normMax[1]{\left|\left|#1\right|\right|_{max}}
\newcommand\normTwo[1]{\left|\left|#1\right|\right|_{2}}
\DeclareMathOperator*{\argmax}{arg\,max}
\DeclareMathOperator*{\argmin}{arg\,min}
\date{}
\title{Breakpoint model}
\hypersetup{
 colorlinks=true,
 pdfauthor={},
 pdftitle={Breakpoint model},
 pdfkeywords={},
 pdfsubject={},
 pdfcreator={Emacs 27.2 (Org mode 9.5.2)},
 pdflang={English}
 }
\begin{document}

\maketitle

\section{Single breakpoint, single slope}
\label{sec:orgec88a0a}

\subsection{Theory}
\label{sec:org338d0d1}

Consider a response variable \(Y\) and an explanatory variable \(X\)
related by:
\begin{align*}
Y = \beta X - \beta (X - \psi)_+ + \varepsilon
\end{align*}

\noindent where \((\beta,\psi) \in \Real^2\), \(\varepsilon
\sim \Gaus[0,\sigma^2]\), and \((x)_+=x\) if \(x>0\) and 0
otherwise. Introduce \(\Theta = (\beta,\psi,\sigma^2)\), we can
express the likelihood relative to \(n\) iid observations as:
\begin{align*}
\Likelihood(\Theta) = \prod_{i=1}^n \frac{1}{\sqrt{2\pi\sigma^2}}\exp\left(-\frac{(Y_i - \beta X_i + \beta (X_i - \psi)_+)^2}{2\sigma^2}\right)
\end{align*}
and the log-likelihood as:
\begin{align*}
\likelihood(\Theta) = - \frac{n}{2} \log(2\pi) - \frac{n}{2} \log(\sigma^2) - \sum_{i=1}^n \frac{\left(Y_i - \beta X_i + \beta (X_i - \psi)_+\right)^2}{2\sigma^2}
\end{align*}

\noindent Maximizing the likelihood with respect to \(\Theta_\mu =
(\beta,\psi)\) is equivalent to minimizing the mean squared
error:
\begin{align*}
\likelihood(\Theta_\mu) = \sum_{i=1}^n (Y_i - \beta X_i + \beta (X_i - \psi)_+)^2
\end{align*}

\noindent which is equivalent to first minimizing
w.r.t. \(beta\), i.e. plug-in the OLS estimator:
\begin{align*}
\tilde{\beta}(\psi) &= \frac{\sum_{i=1}^n X_i Y_i - (X_i - \psi)_+ Y_i}{\sum_{i=1}^n (X_i + (X_i - \psi)_+)^2} = \frac{\sum_{i=1}^n \Ind[X_i \leq \psi] X_i Y_i}{\sum_{i=1}^n \Ind[X_i \leq \psi] X^2_i} \\
\end{align*}

and then minimize w.r.t. \(\psi\):
\begin{align*}
\likelihood(\psi) &= \sum_{i=1}^n (Y_i - \tilde{\beta}(\psi) X_i + \tilde{\beta}(\psi) (X_i - \psi)_+)^2
\end{align*}

So the first 'derivative' should solve:
\begin{align*}
0 =& -2 \sum_{i=1}^n \left[
  \frac{\partial \tilde{\beta}(\psi)}{\partial \psi} (X_i - (X_i - \psi)_+)
- \tilde{\beta}(\psi)\frac{\partial (X_i - \psi)_+}{\partial \psi}\right]
(Y_i - \tilde{\beta}(\psi) X_i + \tilde{\beta}(\psi)(X_i - \psi)_+ ) \\
0 =& \frac{\partial \tilde{\beta}(\psi)}{\partial \psi} \sum_{i=1}^n (X_i - (X_i - \psi)_+)(Y_i - \tilde{\beta}(\psi) X_i + \tilde{\beta}(\psi)(X_i - \psi)_+ ) \\
& - \tilde{\beta}(\psi)\sum_{i=1}^n\frac{\partial (X_i - \psi)_+}{\partial \psi}(Y_i - \tilde{\beta}(\psi) X_i + \tilde{\beta}(\psi)(X_i - \psi)_+ ) 
\end{align*}

Assuming that the breakpoint does not coincide with any datapoint:
\begin{align*}
0 =& - \tilde{\beta}(\psi)\sum_{i=1}^n \Ind[X_i \geq \psi] (Y_i - \tilde{\beta}(\psi) \psi)  \\
\frac{1}{n}\sum_{i=1}^n \Ind[X_i \geq \psi] Y_i =& \tilde{\beta}(\psi) \psi
\end{align*}

So the breakpoint should be such that the average post-breakpoint value equal the fitted plateau.

\clearpage

\subsection{Example}
\label{sec:org70a8a21}

\lstset{language=r,label= ,caption= ,captionpos=b,numbers=none}
\begin{lstlisting}
library(lmbreak)

set.seed(10)
df10 <- simBreak(c(1, 100), breakpoint = c(0,2,4),
                 slope = c(1,0), sigma = 0.05)
e.lmbreak10 <- lmbreak(Y ~ 0 + bp(X, pattern = "10", start = c(2)),
                       data = df11)
model.tables(e.lmbreak10)
\end{lstlisting}

\begin{verbatim}
           X duration intercept    slope
1 0.05773562 1.940351  0.000000 1.003169
2 1.99808692 1.847123  1.946501 0.000000
3 3.84520966       NA  1.946501       NA
\end{verbatim}


OLS:
\lstset{language=r,label= ,caption= ,captionpos=b,numbers=none}
\begin{lstlisting}
XX <- df10$X - pmax(df10$X-coef(e.lmbreak10),0)
solve(t(XX) %*% XX) %*% t(XX) %*% df11$Y
\end{lstlisting}

\begin{verbatim}
         [,1]
[1,] 1.003168
\end{verbatim}


Explicit OLS:
\lstset{language=r,label= ,caption= ,captionpos=b,numbers=none}
\begin{lstlisting}
sum((df10$X < coef(e.lmbreak10))*df10$X*df10$Y)/sum((df10$X < coef(e.lmbreak10))*df10$X^2)
\end{lstlisting}

\begin{verbatim}
[1] 1.003169
\end{verbatim}


Full likelihood:
\lstset{language=r,label= ,caption= ,captionpos=b,numbers=none}
\begin{lstlisting}
calcLogLik <- function(theta){
  beta <- theta["beta"]
  gamma <- theta["gamma"]
  psi <- theta["psi"]
  sigma <- theta["sigma"]
  as.double(-NROW(df11)/2 * log(2*pi) - NROW(df11)/2 * log(sigma) - sum((df11$Y - beta * df11$X - gamma * pmax(df11$X - psi,0))^2)/(2*sigma))
}
theta <- c(beta = unname(coef(e.lmbreak11$model)["Us0"]),
           gamma = unname(coef(e.lmbreak11$model)["Us1"]),
           psi = 1.98454227,
           sigma = sigma(e.lmbreak11$model)^2)
calcLogLik(theta = theta)
logLik(e.lmbreak11$model)
\end{lstlisting}

\begin{verbatim}
[1] 162.2538
'log Lik.' 162.2768 (df=4)
\end{verbatim}


Profile likelihood:
\lstset{language=r,label= ,caption= ,captionpos=b,numbers=none}
\begin{lstlisting}
calcProfLik <- function(psi){ ## psi <- 1
  XX <- cbind(df11$X, pmax(df11$X-psi,0))
  OLS <- as.double(solve(t(XX) %*% XX) %*% t(XX) %*% df11$Y)
  sigma <- sum((df11$Y - XX %*% OLS)^2)/(NROW(df11)-3)
   - NROW(df11)/2 * log(2*pi) - NROW(df11)/2 * log(sigma) - (NROW(df11)-3)/2
}
calcProfLik(theta["psi"])

df.gridPsi <- data.frame(psi = seq(0.5,3.5,length.out=1000))
df.gridPsi$logLik <- sapply(df.gridPsi$psi,calcProfLik)
ggplot(df.gridPsi, aes(x=psi,y=logLik)) + geom_line()
\end{lstlisting}

\begin{verbatim}
[1] 162.2538
\end{verbatim}



Score:
\lstset{language=r,label= ,caption= ,captionpos=b,numbers=none}
\begin{lstlisting}
library(numDeriv)
jacobian(calcLogLik, theta)
jacobian(calcProfLik, theta["psi"])
\end{lstlisting}
\begin{verbatim}
           [,1]        [,2]        [,3]      [,4]
[1,] 0.02242403 0.006498193 0.007915766 -638.0514
            [,1]
[1,] 0.001578588
\end{verbatim}


Hessian:
\lstset{language=r,label= ,caption= ,captionpos=b,numbers=none}
\begin{lstlisting}
Hall <- hessian(calcLogLik, theta)
Iall <- solve(-Hall)

Iall[3,3] - Iall[3,1:2] %*% solve(Iall[1:2,1:2]) %*% Iall[1:2,3]

Hpsi <- hessian(calcProfLik, theta["psi"])
Ipsi <- solve(-Hpsi)

Ipsi/(Iall[3,3] - Iall[3,1:2] %*% solve(Iall[1:2,1:2]) %*% Iall[1:2,3])

\end{lstlisting}

\begin{verbatim}
             [,1]
[1,] 5.740945e-05
         [,1]
[1,] 4.928297
\end{verbatim}


\clearpage

\section{Single breakpoint}
\label{sec:orgf7a689f}

\subsection{Theory}
\label{sec:org24f261c}

Consider a response variable \(Y\) and an explanatory variable \(X\)
related by:
\begin{align*}
Y = \beta X + \gamma (X - \psi)_+ + \varepsilon
\end{align*}

\noindent where \((\beta,\gamma,\psi) \in \Real^3\), \(\varepsilon
\sim \Gaus[0,\sigma^2]\), and \((x)_+=x\) if \(x>0\) and 0
otherwise. Introduce \(\Theta = (\beta,\gamma,\psi,\sigma^2)\), we can
express the likelihood relative to \(n\) iid observations as:
\begin{align*}
\Likelihood(\Theta) = \prod_{i=1}^n \frac{1}{\sqrt{2\pi\sigma^2}}\exp\left(-\frac{(Y_i - \beta X_i - \gamma (X_i - \psi)_+)^2}{2\sigma^2}\right)
\end{align*}
and the log-likelihood as:
\begin{align*}
\likelihood(\Theta) = - \frac{n}{2} \log(2\pi) - \frac{n}{2} \log(\sigma^2) - \sum_{i=1}^n \frac{\left(Y_i - \beta X_i - \gamma (X_i - \psi)_+\right)^2}{2\sigma^2}
\end{align*}

\noindent Maximizing the likelihood with respect to \(\Theta_\mu =
(\beta,\gamma,\psi)\) is equivalent to minimizing the mean squared
error:
\begin{align*}
\likelihood(\Theta_\mu) = \sum_{i=1}^n (Y_i - \beta X_i - \gamma (X_i - \psi)_+)^2
\end{align*}

\noindent which is equivalent to first minimizing
w.r.t. \((\beta,\gamma)\), i.e. plug-in the OLS estimator:
\begin{align*}
(\widehat{\beta},\widehat{\gamma}) = \left(\begin{bmatrix} X \\ (X - \psi)_+ \end{bmatrix} \begin{bmatrix} X & (X - \psi)_+ \end{bmatrix} \right)^{-1} \begin{bmatrix} X \\ (X - \psi)_+ \end{bmatrix} Y \\
= \begin{bmatrix} \trans{X}X & \trans{X}(X - \psi)_+ \\  \trans{X}(X - \psi)_+ & \trans{(X - \psi)_+} (X - \psi)_+ \end{bmatrix}^{-1} \begin{bmatrix} \trans{X} Y \\ \trans{(X - \psi)_+} Y \end{bmatrix}  \\
= \frac{\begin{bmatrix} \trans{(X - \psi)_+} (X - \psi)_+ & -\trans{X}(X - \psi)_+ \\  -\trans{X}(X - \psi)_+ & \trans{X}X \end{bmatrix} \begin{bmatrix} \trans{X} Y \\ \trans{(X - \psi)_+} Y \end{bmatrix} }{\trans{X}X\trans{(X - \psi)_+} (X - \psi)_+ -\trans{X}(X - \psi)_+\trans{X}(X - \psi)_+} 
\end{align*}

where \(Y = (Y_1,\ldots,Y_n)\) and \(X=(X_1,\ldots,X_n)\). We therefore obtain:
\begin{align*}
\tilde{\beta}(\psi) &= \frac{ \trans{(X - \psi)_+} (X - \psi)_+ \trans{X} Y -\trans{X}(X - \psi)_+ \trans{(X - \psi)_+} Y }{\trans{X}X\trans{(X - \psi)_+} (X - \psi)_+ -\trans{X}(X - \psi)_+ \trans{X}(X - \psi)_+}  \\
\tilde{\gamma}(\psi) &= \frac{ \trans{X} X \trans{(X - \psi)_+} Y -\trans{X}(X - \psi)_+ \trans{X} Y }{\trans{X}X\trans{(X - \psi)_+} (X - \psi)_+ -\trans{X}(X - \psi)_+ \trans{X}(X - \psi)_+}  
\end{align*}

and then minimize w.r.t. \(\psi\):
\begin{align*}
\likelihood(\psi) &= \sum_{i=1}^n (Y_i - \tilde{\beta}(\psi) X_i - \tilde{\gamma}(\psi) (X_i - \psi)_+)^2
\end{align*}

Its first derivative is:
\begin{align*}
0 &= -2 \sum_{i=1}^n \left[
  \frac{\partial \tilde{\beta}(\psi)}{\partial \psi} X_i
+ \frac{\partial \tilde{\gamma}(\psi)}{\partial \psi}(X_i - \psi)_+
+ \tilde{\gamma}(\psi)\frac{\partial (X_i - \psi)_+}{\partial \psi}\right]
(Y_i - \tilde{\beta}(\psi) X_i - \tilde{\gamma}(\psi) (X_i - \psi)_+)
\end{align*}

\clearpage

\subsection{Example}
\label{sec:org5660144}

\lstset{language=r,label= ,caption= ,captionpos=b,numbers=none}
\begin{lstlisting}
library(lmbreak)

set.seed(10)
df11 <- simBreak(c(1, 100), breakpoint = c(0,2,4),
                 slope = c(1,0), sigma = 0.05)
e.lmbreak11 <- lmbreak(Y ~ 0 + bp(X, pattern = "11", start = c(2)),
                       data = df11)
model.tables(e.lmbreak11)
\end{lstlisting}

\begin{verbatim}
           X duration intercept      slope
1 0.05773562 1.926807  0.000000 1.00316914
2 1.98454227 1.860667  1.932913 0.01677379
3 3.84520966       NA  1.964123         NA
\end{verbatim}


OLS:
\lstset{language=r,label= ,caption= ,captionpos=b,numbers=none}
\begin{lstlisting}
XX <- cbind(df11$X, pmax(df11$X-coef(e.lmbreak11),0))
solve(t(XX) %*% XX) %*% t(XX) %*% df11$Y
\end{lstlisting}

\begin{verbatim}
           [,1]
[1,]  1.0031692
[2,] -0.9863952
\end{verbatim}


Explicit OLS:
\lstset{language=r,label= ,caption= ,captionpos=b,numbers=none}
\begin{lstlisting}
(crossprod(XX[,2]) * crossprod(XX[,1],df11$Y) - crossprod(XX[,1],XX[,2]) * crossprod(XX[,2],df11$Y)) / (crossprod(XX[,1]) * crossprod(XX[,2]) - crossprod(XX[,1],XX[,2])^2)
(crossprod(XX[,1]) * crossprod(XX[,2],df11$Y) - crossprod(XX[,1],XX[,2]) * crossprod(XX[,1],df11$Y)) / (crossprod(XX[,1]) * crossprod(XX[,2]) - crossprod(XX[,1],XX[,2])^2)
\end{lstlisting}

\begin{verbatim}
         [,1]
[1,] 1.003169
           [,1]
[1,] -0.9863952
\end{verbatim}


Full likelihood:
\lstset{language=r,label= ,caption= ,captionpos=b,numbers=none}
\begin{lstlisting}
calcLogLik <- function(theta){
  beta <- theta["beta"]
  gamma <- theta["gamma"]
  psi <- theta["psi"]
  sigma <- theta["sigma"]
  as.double(-NROW(df11)/2 * log(2*pi) - NROW(df11)/2 * log(sigma) - sum((df11$Y - beta * df11$X - gamma * pmax(df11$X - psi,0))^2)/(2*sigma))
}
theta <- c(beta = unname(coef(e.lmbreak11$model)["Us0"]),
           gamma = unname(coef(e.lmbreak11$model)["Us1"]),
           psi = 1.98454227,
           sigma = sigma(e.lmbreak11$model)^2)
calcLogLik(theta = theta)
logLik(e.lmbreak11$model)
\end{lstlisting}

\begin{verbatim}
[1] 162.2538
'log Lik.' 162.2768 (df=4)
\end{verbatim}


Profile likelihood:
\lstset{language=r,label= ,caption= ,captionpos=b,numbers=none}
\begin{lstlisting}
calcProfLik <- function(psi){ ## psi <- 1
  XX <- cbind(df11$X, pmax(df11$X-psi,0))
  OLS <- as.double(solve(t(XX) %*% XX) %*% t(XX) %*% df11$Y)
  sigma <- sum((df11$Y - XX %*% OLS)^2)/(NROW(df11)-3)
   - NROW(df11)/2 * log(2*pi) - NROW(df11)/2 * log(sigma) - (NROW(df11)-3)/2
}
calcProfLik(theta["psi"])

df.gridPsi <- data.frame(psi = seq(0.5,3.5,length.out=1000))
df.gridPsi$logLik <- sapply(df.gridPsi$psi,calcProfLik)
ggplot(df.gridPsi, aes(x=psi,y=logLik)) + geom_line()
\end{lstlisting}

\begin{verbatim}
[1] 162.2538
\end{verbatim}



Score:
\lstset{language=r,label= ,caption= ,captionpos=b,numbers=none}
\begin{lstlisting}
library(numDeriv)
jacobian(calcLogLik, theta)
jacobian(calcProfLik, theta["psi"])
\end{lstlisting}
\begin{verbatim}
           [,1]        [,2]        [,3]      [,4]
[1,] 0.02242403 0.006498193 0.007915766 -638.0514
            [,1]
[1,] 0.001578588
\end{verbatim}


Hessian:
\lstset{language=r,label= ,caption= ,captionpos=b,numbers=none}
\begin{lstlisting}
Hall <- hessian(calcLogLik, theta)
Iall <- solve(-Hall)

Iall[3,3] - Iall[3,1:2] %*% solve(Iall[1:2,1:2]) %*% Iall[1:2,3]

Hpsi <- hessian(calcProfLik, theta["psi"])
Ipsi <- solve(-Hpsi)

Ipsi/(Iall[3,3] - Iall[3,1:2] %*% solve(Iall[1:2,1:2]) %*% Iall[1:2,3])

\end{lstlisting}

\begin{verbatim}
             [,1]
[1,] 5.740945e-05
         [,1]
[1,] 4.928297
\end{verbatim}


\clearpage

\section{Multiple breakpoints}
\label{sec:orga5dc241}

We now consider the more general case where:
\begin{align*}
Y = \beta X(\psi) + \varepsilon
\end{align*}
where \(\beta\) is a vector of coefficients and \(X(\psi)\) the design
matrix depending on a vector of breakpoint \(\psi\). Similarly to the
previous derivations we need to minimize the mean square loss:
\begin{align*}
\sum_{i=1}^n \left(Y_i - \beta X_i(\psi)\right)^2
\end{align*}

with respect to \(\beta\) and \(\psi\). For given \(\psi\) the
coefficient \(\beta\) minimizing this loss are given by the OLS
estimator:
\begin{align*}
\widehat{\beta} = (\trans{X}(\psi)X(\psi))^{-1}\trans{X}(\psi) Y
\end{align*}

\clearpage

\section{Proximal gradient method}
\label{sec:org8759e44}

One difficulty is that this objective function is not differientiable
in \(\psi\) at \(\left(X_i\right)_{i=1}^n\). 

\bigskip

\(\likelihood(\Theta_\mu)\) might not be strictly convex but it is
convex. So we can try applying a proximal gradient algorithm. This
means updating the estimate by:
\begin{align*}
\Theta_{\mu,k+1} &= \text{prox}_{\alpha_k \likelihood}(\Theta_{\mu,k}) = \argmin_{\Theta_\mu \in \Real^2} \left( \likelihood(\Theta_\mu) + \frac{1}{2\alpha_k}||\Theta_\mu-\Theta_{\mu,k}||^2 \right) \\
&= \argmin_{\Theta_\mu \in \Real^3} \left( \sum_{i=1} (Y_i - \beta X_i - \gamma (X_i - \psi)_+)^2 + \frac{(\beta - \beta_k)^2+(\gamma - \gamma_k)^2+(\psi - \psi_k)^2}{2\alpha_k} \right) 
&= \argmin_{\psi \in \Real} \left((I - Z)(\trans{Z}Z)^{-1}ZY + \frac{(\beta - \beta_k)^2+(\gamma - \gamma_k)^2+(\psi - \psi_k)^2}{2\alpha_k}  \right)
\end{align*}
where \(\alpha_k\) is a pre-defined stricly positive real value.
\end{document}